\chapter{全文总结}
\label{chapter:conclusion}

\section{主要结论}
网络功能虚拟化为传统电信业务描绘了十分美好的前景,是对已有电信架构的一次革新,大大提升了电信业务的开发效率,缩短了开发周期,降低了部署的成本。通过与现有的云计算基础设施相结合,网络功能虚拟化将实现服务的快速部署、伸缩和升级。然而,在从专有硬件向通用服务器迁移,软件化网络功能的过程中,不可避免地会遇到性能和服务质量下降的问题。特别是考虑到现有的大容量通用服务器中,多核的非一致性存储架构是被采用的主流架构,虚拟化的物理资源之间存在亲和度差异,不同亲和度的物理资源(CPU,内存等)之间的数据访问也会有较大的性能差异。这样的底层物理特性结合网络功能虚拟化中特殊的服务链结构,会引起很高的性能波动,导致服务质量下降。

本设计的研究内容主要着眼于通用的多核服务器中网络功能服务链的资源映射问题。在由单核虚拟机所组成的服务节点集群中,上层应用对特定网络服务的需求仅仅以服务链的方式下发,对于具体组成服务链的物理实例的资源分布并不知晓,而在多核的非一致性存储访问平台下,被分配不同物理资源的虚拟机彼此之间具有不同的资源亲和度,即实例虚拟机间访问性能与其物理资源的数据传输性能。这种底层资源的亲和度关系对于上层应用是透明的,也就是说上层服务在组织时忽视了到这层底层信息。因此在组织服务链提供服务时,可能会出现较大的性能波动。然而,正是多核架构存在的这种亲和度关系,给组织网络功能服务链时提供了新的解决思路。

基于以上的考虑,本设计中利用多核物理核的非一致性存储访问和虚拟化相关技术,在IMS服务平台下对服务链的资源映射问题进行了研究。当服务请求被发出时,通过解析请求,获取实际服务链的组成情况,结合底层资源亲和度的采样信息,采用基于贪心的搜索方法,并同时考虑当前的运行负载,避免出现局部过热或资源竞争严重的不利情况,构造出理论上亲和度较好的服务链映射策略。之后将该策略应用到实际服务链的组链中,达到优化服务链性能的目的。

最后,本文在实际的大容量通用服务器上,利用真实的网络服务Clearwater来验证了本设计的实现原型,并在真实的网络环境中对本设计进行了性能的评估。数据分析的结果表明,本设计在实验环境中可以平均可以减少40\%具体IMS业务服务延迟;并且在网络性能测试中,在模拟服务链应用测试下,分别有最多30\%的延迟下降和最高接近2倍的带宽提升。这样的实验数据可以说明在使用相同数量的物理资源前提下,本文的设计可以提高网络服务的性能。

\section{研究展望}
本设计虽然已经在通用的服务器上实现了初级原型并获得了一定了性能提升,但仍然存在一些可以提升和改进的空间。首先,本设计从单台服务器的物理资源亲和度角度出发,而在实际生产环境中,尤其是云计算数据中心中,机架上的服务器都是以高性能网卡相连所组成的服务器集群,本文的资源映射模型可以扩展到多台物理机的范围。另外本文所提出的映射算法在问题规模较小的前提下具有很好的效果,但是当问题规模扩展到一定级别,算法的执行效率会大大下降,这时候需要结合一些已有的高级算法来提升算法的时间消耗以满足实际生产中的响应需求。

希望本文所完成的工作能够具有一些抛砖引玉的积极作用,为其他研究者在网络功能虚拟化的研究提供有益的帮助和参考。
