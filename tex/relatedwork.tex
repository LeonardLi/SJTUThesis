\chapter{网络功能虚拟化相关技术}
\label{chap:relatedwork}

\section{网络功能虚拟化发展概况}
%ETSI官方背景
欧洲电信标准化协会(ETSI)作为NFV的发起标准组织,自2013年开始陆续发布了NFV参考架构等系列白皮书。虽然ETSI NFV工作小组发布的各项白皮书不是强制执行的标准,但是得到了业界的普遍认可,已经成为了业界的事实标准。目前NFV的标准框架已得到各方的基本认可,各厂商也基本按照现有的标准架构开发和推进各自的服务框架。如图 \ref{fig:NFV} 所示,NFV标准框架主要由NFV基础设施(NFV Infrastructure,NFVI)、虚拟网络功能(Virtual Network Functions, VNFs)和NFV管理与编排系统(NFV Management and Orchestration, NFV MANO)三个主要部分组成。其中,NFV基础设施指的是承载具体业务的通过虚拟化技术所管理的数据中心资源。与传统电信机房不同,数据中心具有统一的管理标准和更加普通的 x86 通用服务器,其维护管理成本和业务升级的成本相较与传统电信业务的中心机房 (Central Office) 要更加低廉。虚拟网络功能是具体业务的运行实体,结合传统电信的业务支撑系统和管理支撑系统 (OSS/BSS) 作为上层应用系统,以虚拟化软件的形式,运行在NFV基础设施中,并接受管理编排系统的功能管理。而NFV管理和编排子系统作为衔接IT虚拟化基础设施和传统电信业务的关键系统,受到了众多厂家的重点关注,也是目前研究的重点之一。MANO系统负责解析上层业务、映射并管理具体的物理资源、编排具体业务系统等多项重要业务,其重要性不言而喻。需要特别指出的是,底层资源即通用的大容量服务器资源的管理对于传统电信服务来说是全新的问题和挑战。

\begin{figure}[!htp]
	\centering
	\includegraphics[width=1\textwidth]{ETSI.png}
	\bicaption[fig:NFV]{NFV标准框架}{NFV标准框架 \citen{etsi2013001}}{Fig}{NFV Standard Architecture\citen{etsi2013001}}
\end{figure}

自白皮书发表以来,ETSI的NFV规范小组制定了每两年为一个发布周期的工作计划,持续推出更加具体的协议规范和标准术语定义以及NFV的使用案例。截止本文,NFV小组已经进入了第三个发布周期。

在第一个周期2013-2014年内,小组的主要工作目标是:
\begin{enumerate*}[label=\itshape\alph*)\upshape]
	\item 推动网络运营商融合NFV;
	\item 将已有的适用标准纳入网络服务和产品中;
	\item 以促进创新和培育供应商开放的生态系统为目标,同步地开发新的技术规范。
\end{enumerate*}

截止2014年,工作组除了前期发布的6篇定义性文档,又陆续发布了11篇详细的技术规范。第一套文档标志着NFV的标准化初期工作已经完成,需要对划分的各个工作领域进行进一步的详细定义和约束。第二个周期2015-2016年,小组的工作重心已经转移到了VIM,VNFM,NFVO以及各功能接口的具体定义中。进一步的,在该阶段的工作中,小组还重点推出了适用于当前NFV服务需求的信息模型参考,在模型中预留了解决各项实际问题的参数接口,为各厂商开发自己的NFV服务框架提供指导。各项文档均已在ETSI官方网站发布。目前,NFV规范小组正处在2017-2018发布周期内。根据已发布的计划,在本周期内他们有23个工作目标,包括20个新功能点和3个对上一周期的已发布文档的补充。在该周期内,小组的工作重点是深入定义核心的NFV信息模型,完善支持NFV非业务子系统接口如计费,账单,用户管理,监管,安全等,以及对MANO服务的多方面支持。同时在近些年来,NFV工作组联合了如 Linux Foundation,OpenDayLight 等开源社区以及工业界各服务提供厂商,旨在携手推进NFV落地实现,与现有业务平台平滑过渡衔接。他们推出了各种开源的NFV平台如 OPNFV,OpenSource MANO,Open-O, Cloudify等等。

在本文中,我们将Clearwater服务作为NFV标准框架中的上层业务系统,使用基于KVM的虚拟化平台和Virtio的I/O虚拟化方式来将该服务部署在标准的NUMA架构通用服务器中,作为网络功能虚拟化的基础设施(NFVI)。在Clearwater服务中,各个功能节点实例以虚拟机的形式运行在通用服务器中,本文所提出的优化映射方法主要与NFV管理与编排系统(MANO)中映射和管理物理资源功能相对应,目的是为了能够有效地提高在非一致性存储访问架构服务器中物理资源的使用效率。

\section{Clearwater平台介绍}
\label{intro:clearwater}
目前全球的运营开发商都在加速商用的VoLTE (Voice over Long-Term Evolution) 开发, IP多媒体子系统 (IP Multimedia Subsystem, IMS)是4G VoLTE商用组件之一,因此也引起了工业界格外的重视。NFV兴起以来,工业界纷纷在尝试将IMS系统部署在NFV平台环境中,探索IMS应用场景在NFV生产环境中的实现,希望能够利用虚拟化技术加速电信业务的研发和部署,降低业务升级的成本和周期。根据本文的调研,在开源的IMS系统领域,有两个十分知名的项目:Clearwater \citen{clearwater}与 Kamalio \citen{kamalio}。而在这两者之中,Clearwater系统更是因为其为云计算平台量身打造的多媒体子系统定位而广受众多厂商和研发人员的欢迎,开发社区十分火热,产品也在迅速的迭代中。它采用大型互联网软件架构的设计思路,以微服务的方式设计各个组件,使得系统本身具有很好的伸缩能力,成为了一个商用级别的开源IMS系统。
\begin{figure}[!htp]
	\centering
	\includegraphics[width=0.5\textwidth]{ims.jpg}
	\bicaption[fig:ims]{IMS系统架构图}{IMS系统架构图}{Fig}{IMS Architecture}
\end{figure}

IMS服务是由一系统功能组组成的,各个功能组之间由一组标准接口联链起来,组成一个IMS管理网络,其架构如图 \ref{fig:ims}所示。一个功能组并非是一个节点,它的实现方式是开放的,允许将多个功能组布署在一个节点,同时也允许一个功能组由多个节点实现。考虑到容量、负载均衡和管理等方面,IMS服务还允许在一个网络存在有多个相同的功能组。在IMS标准中,其核心网络功能组包括:归属用户服务器(Home Subscriber Server, HSS),呼叫会话控制功能(Call Session Control Function, CSCF),应用服务器(Application Server, AS),出口网关控制功能(Breakout Gateway Control Function, BGCF),媒体网关控制功能(Media Gateway Control Function, MGCF)等等 \footnote{http://www.3gpp.org/technologies/keywords-acronyms/109-ims}。这些核心网络功能组的功能介绍如表\ref{tab:IMS}所示 。

\begin{table}
	\centering
	\bicaption[tab:IMS]{IMS服务核心网络功能组}{IMS服务核心网络功能组}{Table}{IMS Core components}	
	\begin{tabular}{| l | p{9cm} |}  \hline
		名称 & 功能介绍 \\ \hline
	\textbf{归属用户服务器(HSS)}	 & 是一个核心的用户数据库,它为IMS网络中实际管理通话的实体提供支持。如存取用户相关的信息即用户配置,对用户认证和授权以及提供用户位置IP地址等相关信息。	\\ \hline
	\textbf{呼叫会话控制功能(CSCF)}	& 由会话发起协议(Session Initiation Protocol, SIP)服务器和代理共同实现通话控制功能,在IMS系统中负责处理SIP信号数据包。	\\ \hline
	\textbf{SIP应用服务器(AS)} &  负责使用SIP协议与CSCF之间通信并提供HSS服务查询接口。  	\\ \hline
	\textbf{媒体网关控制功能(MGCF)} &负责完成IMS网络与PSTN网络之间的呼叫控制协议转换。其主要功能是将SIP消息进行转换,并管理PSTN (public switched telephone network) 网络的负载以及与IMS网络IP流间的连接。 		\\ \hline
	\textbf{出口网关控制功能(BGCF)} & 是一个SIP代理,它处理来自CSCF的路由请求。BGCF有基于电话号码的路由功能,用来选择与PSTN网络的接口点。当BGCF发现被叫网络位于一个PSTN网络时,BGCF就会选择一个媒体网关控制功能(MGCF),将会话路由到MGCF服务,由MGCF服务负责与PSTN网络交互。	 	\\ \hline
	\end{tabular}				
\end{table}

\begin{figure}[!htp]
	\centering
	\includegraphics[width=1.0\textwidth]{clearwater/clearwater_arch2.png}
	\bicaption[fig:clearwater]{Clearwater系统架构图}{Clearwater系统架构图\citen{clearwater}}{Fig}{Clearwater Architecture\citen{clearwater}}
\end{figure}
本文所参考的Clearwater分布式版本的系统架构图如图 \ref{fig:clearwater} 所示,可以看出该系统主要有五个核心的组件:Bono,Sprout,Dime,Vellum和Homer。分别对应实现了IMS标准服务中的各项功能。其中,Bono节点作为SIP的边界代理,负责承担IMS服务的接入业务,通过WebRTC接口为客户端提供服务接入点。Sprout节点作为注册和权限管理的路由代理,主要负责处理客户端的认证和应用服务器ISC接口的通信。Sprout节点还包含内置的MMTEL(multimedia telephony)应用服务器,在运行过程中Sprout不负责存储任何持久化数据,而是通过HTTP协议从Dime和Homer服务中读取HSS的配置信息,如用户信息和MMTEL服务配置等。Sprout还通过远程调用接口从Vellum中获取用户的注册信息和当前状态。Dime节点主要负责运行Homestead和Ralf服务,包括存储HSS信息的缓存等具体功能。Homer是一个标准的用户设置存储服务,通过标准的接口来增删改查用户持久化数据。而Vellum则负责存储各种持久化信息,包括用户的注册信息,当前服务的始终状态以及各种权限和密钥等。Ellis作为一个前端管理工具,并不参与IMS实际业务,只是官方提供的一个用于管理整个平台的可视化工具原型。

Clearwater是针对云计算数据中心开发的IMS系统。众多大型电信运营商纷纷采用基于IMS的标准架构作为其IP语音通话、视频和消息服务的基础构建,用以取代基于传统电路交换系统的上一代VoIP业务系统。Clearwater遵循IMS架构原则,实现了IMS核心网络所需的所有关键标准化接口。有别于别的IMS实现,Clearwater是以虚拟化云环境作为基础设施的角度设计的。Clearwater通过结合已在许多Web应用程序中验证过的设计模式和开源软件组件,实现了前所未有的大规模可扩展性和卓越的成本效益。正因为有这样的特点,本文选取了Clearwater作为NFV应用平台,在其基础上测试本文所提出的优化系统原型。

\section{常见的网络I/O虚拟化技术}
\label{intro:IO}
虚拟化技术作为NFV基础,是实现NFV的关键技术。I/O虚拟化,特别是网络I/O虚拟化,更是直接影响NFV性能的关键因素。在本节中,将介绍和分析三种主流的网络虚拟化模型:设备仿真模型(Device Emulation),分离驱动程序模型(Split-driver)和硬件辅助模型(Hardware-assisted)。本文实现的系统所采用的I/O虚拟化方式为Virtio\citen{russell2008virtio,kivity2007kvm},接下来将介绍这三种I/O虚拟化方式的原理,并从每种模型的原理阐述选择Virtio作为I/O虚拟化方式的优势。

\subsection{设备模拟模式}
设备模拟模式是一种通过在虚拟机监视器中模拟敏感指令和特权指令的方式来模拟物理设备I/O操作的全虚拟化实现方式。如图 \ref{fig:device_emu} 所示。这种模式充分利用二进制翻译技术和直接执行技术的组合来从客户操作系统中抽象和分离底层硬件,实现纯软件方式的硬件仿真。虚拟机操作系统实例和用户级别指令可以在没有感知底层虚拟化环境的情况下不加修改地运行,所以设备模拟模式为I/O操作提供了最佳性能隔离和安全性,简化了客户端虚拟机的迁移操作和提升了其可移植性。工业界有很多成熟的产品如VMware Workstation \citen{sugerman2001virtualizing} 和VMware ESX Server \citen{ahmad2003analysis} 都支持这种I/O虚拟化的方式。此外,XEN和KVM的QEMU也可以配置支持设备模拟模式的I/O虚拟化。
\begin{figure}[!htp]
	\centering
	\includegraphics[width=0.8\textwidth]{Device_Emulation2.png}
	\bicaption[fig:device_emu]{设备模拟模式示意图}{设备模拟模式示意图}{Fig}{Device Emulation Model Overview}
\end{figure}
但是,这种方法也存在很大的缺陷,由于虚拟机监视器需要动态翻译所有操作系统指令并缓存所有结果,因此虚拟机监视器设备驱动程序一旦发生故障将导致所有客户虚拟机的I/O中断。并且,由于在运行时对这些敏感和特权的指令请求处理需要进行代价高昂的陷入和仿真操作,频繁的此类操作将引起巨大的性能开销,成为影响I/O性能的瓶颈。根据已有的研究,每个I/O操作的陷入和仿真都将导致客户操作系统和虚拟机监视器之间的上下文切换,其成本约为3000到5000个CPU周期 \citen{dong2011optimizing}。根据已有的研究实验显示,这样的开销显着地降低了近10倍的I/O吞吐量。这一性能瓶颈挑战严重阻碍了将设备仿真方法部署到高性能网络和配备现代高性能网卡(例如40/100 Gbps 网卡)的数据中心中。NFV的应用场景下需要数量众多的虚拟机频繁的使用各自的网络I/O设备,设备模拟的方式并不能胜任该场景下对I/O设备高性能的需求。

\subsection{分离驱动模式}
为了缓解虚拟机监视器参与模拟的高昂开销,Xen团队首先在半虚拟化场景下开发了分离驱动模式(Split-Driver) \citen{barham2003xen} 。紧随其后,KVM的virtio \citen{russell2008virtio,kivity2007kvm} ,VMware tool和VM interface \citen{amsden2006vmi} 也支持了在半虚拟化的分离驱动模式。本文以virtio为例,介绍了在KVM平台下奋力驱动模式的实现原理。KVM的分离驱动模型如图 \ref{fig:split_dri} 所示,它由主机中的前端virtio驱动和后端驱动组成。当虚拟机向服务器内的另一台虚拟机发送数据时,虚拟机首先将数据加载到其vring缓冲区中的队列,并修改其描述符表。随后通知KVM产生vm-exit信号,并使用ioeventfd方法允许vHost驱动程序接收来自客户端虚拟机的信号。
\begin{figure}[!htp]
	\centering
	\includegraphics[width=0.8\textwidth]{Split_driver2.png}
	\bicaption[fig:split_dri]{Split-driver模式示意图}{Split-driver模式示意图}{Fig}{Split-driver Model Overview}
\end{figure}
当vHost驱动获得信号后,前端程序将获取有效队列的物理地址,并将需要传输的数据复制到绑定tap设备地址空间。数据通过网络协议栈被传送到目的地址上。在传送过程中,前端驱动发现如果目标地址是同一台机器上的虚拟机,vHost将利用零拷贝机制,通过共享内存直接将数据的地址指针发送到到目标虚拟机的对应驱动中。当目标虚拟机收到数据后会触发一个回调信号函数,并让虚拟机将传输的数据拷贝放入自己的vring读取队列。总而言之,在分离驱动模式下的虚拟机之间的数据传输可以被看作是将数据从一个内存区域复制到另一块存储区域。分离驱动模式采用高效的I/O通信通道 (I/O Channel) 来避免虚拟机监视器对模拟端口I/O和内存映射功能的干预。因此,分离驱动模式可以利用同样的物理资源达到比设备仿真模式下更高的吞吐量,同时降低CPU的利用率。即使半虚拟化方式需要对操作系统内核进行深度修改,无法做到像全虚拟化方式那样对不同客户机操作系统广泛的支持,分离驱动模式还是显示出了其在虚拟机管理和虚拟机迁移方面的巨大灵活性,在当下的虚拟机监视器和虚拟机操作系统中得到了广泛的支持。


\subsection{硬件辅助模式}
设备仿真和分离驱动模式都被分类为基于软件的I/O虚拟化,这种基于软件的模式实现了丰富的I/O设备功能同时简化了对虚拟机I/O设备的管理。然而,这种方式也并不是完美无缺的,纯软件实现方式在面对大量的陷入-仿真操作或者大块数据拷贝的时候会面临高昂的系统开销,这个时候可以借助硬件辅助的模式来解决这种情况下的瓶颈。硬件辅助I/O虚拟化模式是为共享本地设备而开发的,它通过继承I/O内存管理单元 (IOMMU) 的用户权限来使用直接I/O技术从而实现绕开虚拟化的内存保护和地址转换,为虚拟机直接分配真实的I/O设备。PCI-SIG(Peripheral Component Interconnect Special Interest Group, 周边元件互连特别兴趣小组)工作组针对这种I/O虚拟化的方式制定了规范,通过为每个虚拟机提供独立的内存空间,中断和DMA流来绕过虚拟机监视器直接参与数据处理。特别的,PCI-SIG推出了单根I/O虚拟化 (SR-IOV) 技术,这项技术为针对PCIe设备设计了一组硬件增强技术,通过去除虚拟机监视器在数据包分类和内存地址翻译的干预来实现基于专门硬件的性能提高。
\begin{figure}[!htp]
	\centering
	\includegraphics[width=0.7\textwidth]{SR_IOV2.png}
	\bicaption[fig:hardware_ass]{硬件辅助模式示意图}{硬件辅助模式示意图}{Fig}{Hardware-assisted Model Overview}
\end{figure}
SR-IOV虚拟化模式结构如图 \ref{fig:hardware_ass} 所示,一个支持SR-IOV功能的设备可以由虚拟机监视器分配多个虚拟设备(VF, Virtual Function),以普通PCI设备的身份被分配PCI地址空间。宿主机中的物理设备 (PF, Physical Function) 的驱动程序负责管理和配置VF,而在被分配VF的虚拟机操作系统中,驱动程序将其当做普通的直接分配的物理设备来使用,通过IOMMU直接访问自己专有的VF。通过这样的方式,SR-IOV可以绕过虚拟机监视器实现数据的传递,从而减少移动数据的开销,在降低CPU利用率的同时,减少系统的延迟并提高网络吞吐量。但是需要指出的是,这种硬件辅助的方式需要手动进行虚拟网络功能的配置,无法实现像软件实现方式那样灵活的迁移和扩展。并且一块PF所能分配的VF是有上限的,这也从硬件角度限制了硬件辅助方式的在多虚拟机应用场景下的应用。


\section{非一致性存储访问架构}
\label{sec:numa}
众所周知,NFV的目标是在大容量的标准通用服务器上运行传统通信业务,而现在通用的标准服务器基本都是遵循非一致性存储访问架构 (NUMA, Non-Uniform Memory Access) 的服务器阵列,为了更好地完成NFV到通用服务器上的迁移,了解通用服务器的基础架构十分必要。图 \ref{fig:numa} 所示的是Intel Haswell架构下多核处理器示意图,在NUMA服务器中每个处理节点 (Socket/Die) 通过内存访问控制器 (MC,Memory Controller) 与相邻的内存直接相连。同一个处理节点中多个物理核拥有自己独占的L1/L2级缓存并且通过节点内部链路通道共享L3缓存,内存访问控制器以及PCI-e接口。在一个处理节点内部数据传输是通过点对点的高速数据通路 (例如Intel的QPI) 实现的。除了同一个节点内通信,不同的节点间也由相同的高速数据通路相连。不同的处理器节点要访问别的节点内存时,需要通过内存控制器进行远程访问。
\begin{figure}[!htp]
	\centering
	\includegraphics[width=0.8\textwidth]{numa.pdf}
	\bicaption[fig:numa]{Intel Haswell架构非一致性存储访问示意图}{Intel Haswell架构非一致性存储访问示意图}{Fig}{Intel Haswell NUMA Architecture Overview}
\end{figure}
在NUMA架构下,本地和远程内存访问由于机制的差异存在性能上的差异,而节点内共享的内存控制器在竞争过高的情况下也容易成为性能瓶颈导致内存访问性能的剧烈下降,所以如何针对多核通用服务器下非一致性访问的特性来优化应用一直是学术界所面对的经典挑战之一。NFV应用迁移到多核服务器中,不可避免地会遇到NUMA架构所带来的问题,而SFC的服务组织形式下同一条服务链上的虚拟机彼此之间存在大量的数据拷贝传输,如何处理好基于NUMA物理架构的物理资源对实现高性能的NFV具有重要意义。


\section{多核物理机虚拟化应用性能观测与分析}
\label{related:observe}
在多核物理机中,由于存在非一致性存储访问架构的影响,分配了不同物理资源的两台虚拟机之间的数据传输性能会有较大的差异。在NFV应用的场景下,网络功能单元是以虚拟机的形式存在,节点间的数据传输实质上是虚拟机利用网络虚拟化的技术进行通信。从服务器硬件的角度看,可以当做一个虚拟机线程向另一个虚拟机线程拷贝或者接收数据。那么在这样的前提下,虚拟机线程所分配的物理资源之间的数据传输性能对虚拟机线程间通信的性能会造成影响。为了验证这一想法,本文在一台通用服务器上进行了模拟NFV应用的的测试。测试平台具体参数见实验部分的表 \ref{tab:configure},使用的虚拟化平台为KVM,网络虚拟化方式为Virtio。实验的设置如图 \ref{fig:illustrative} 所示。实验使用两台虚拟机,其中一台固定在一个处理器节点上并绑定本地内存作为虚拟机的分配内存,而另一台虚拟机则不断更改所绑定的物理核及内存,以遍历所有绑定情况作为测试样本,测试的结果如图 \ref{exp:observe} 所示。这里我们选用的基准测试工具Netperf和Httperf如表 \ref{tab:illustrate_setup} 所示,分别模拟微观的数据包负载和宏观的应用负载来测试不同资源组合间虚拟机通信的性能表现。

\begin{figure}[!htp]
	\centering
	\includegraphics[width=0.8\textwidth]{observation/observation_experiment.pdf}
	\bicaption[fig:illustrative]{模拟应用性能测试示意图}{模拟应用性能测试示意图}{Fig}{Illustrative Experiment Setup}
\end{figure}

\begin{figure}[!htp]
	\centering
	\subfigure[Netperf带宽测试]{
		\label{exp:netperf_bandwidth}
		\includegraphics[width=0.8\textwidth]{observation/box1.pdf}
	}
	\subfigure[Httperf带宽测试]{
		\label{exp:httperf}
		\includegraphics[width=0.8\textwidth]{observation/box2.pdf}
	}
	\subfigure[Netperf延迟测试]{
		\label{exp:netperf_latency}
		\includegraphics[width=0.8\textwidth]{observation/box3.pdf}
	}
	\bicaption[exp:observe]{模拟应用性能观测结果}{模拟应用性能观测结果}{Fig}{Application Performance Observation}
\end{figure}

\begin{table}[!htb]
	\centering
	\bicaption[tab:illustrate_setup]{验证性实验基准测试工具}{验证性实验基准测试工具}{Table}{Illustrative Test Benchmarks}
	\begin{tabular}{ | l | p{6cm} |}\hline
		\textbf{基准测试工具集} &							 \textbf{选用的实际负载}  				\\ 	\hline
		Netperf 2.6.0 	&  TCP\_STREAM, TCP\_RR	  \\ \hline
	    Httperf 0.9.0	&  Apache 2.4.7   \\ \hline
	\end{tabular}
\end{table}
\begin{table}
	\centering
	\bicaption[tab:illustrate1]{TCP\_STREAM\_1024性能测试结果数据分析}{TCP\_STREAM\_1024性能测试结果数据分析}{Table}{TCP\_STREAM\_1024 Test Performance Data Analysis }	
	\begin{tabular}{ccc}  \hline
		分组/Gbps & 频率 & 累计百分比 \\ \hline
		[1000-3000) & 	2 & 2.17\% \\ \hline
	    [3000-5000)	& 	15 & 18.47\%  \\ \hline
		[5000-7000)	& 	20 & 40.21\% \\ \hline
		[7000-9000)	&	42 &	85.86\% \\ \hline
		[9000-12000)	&	13 & 100\% \\ 	\hline
	\end{tabular}				
\end{table}
\begin{table}
	\centering
	\bicaption[tab:illustrate2]{HTTP\_2000性能测试结果数据分析}{HTTP\_2000性能测试结果数据分析}{Table}{HTTP\_2000 Test Performance Data Analysis }	
	\begin{tabular}{ccc}  \hline
		 分组/Connections & 频率 & 累计百分比  \\ \hline
		[2800-3000) &7	&7.29\% \\ \hline
		[3000-3200)	&21	&29.17\% \\ \hline
		[3200-3400)	&19	&48.96\% \\ \hline
		[3400-3600)	&36	&86.46\% \\ \hline
		[3600-3800)	&13	&100\% \\ \hline
	\end{tabular}				
\end{table}
\newpage
本文使用盒图的方式来展现模拟应用性能测试观测结果,将所获得的样本数据直观地体现在盒图中。实验分别观测了微观下数据包带宽和延迟以及宏观下网络应用访问连接数。据根据测试结果不难发现,在分配了相同数量的物理资源前提下,盒图中数据盒的宽度较大并且有大量的离群点出现,离群点之间的观测值差异甚至达到了接近三倍的差距如图 2-9(a-c) 所示。这表明在不同的资源组合模式下测试性能结果分布的离散程度较大,即虚拟机之间的数据传输性能在不同的虚拟机资源亲和度下存在较大的波动,具体来看,选取模拟应用测试中Netperf工具中TCP\_Stream负载在数据包大小为1024 bit和Httperf工具在并发数为2000时的测试结果进行数据频率分布统计,统计结果如表 \ref{tab:illustrate1} 和表 \ref{tab:illustrate2} 所示。根据统计结果可以看出,在相同的测试负载下,测试结果在各个分组区间中所占百分比较为接近,数据较为均匀的分布在各个区间中,对于实际应用来说存在提升性能的空间。根据这样的实验观测和分析结果,可以认为针对虚拟机物理资源间的数据传输性能来优化服务链中的资源映射从而提升服务链数据传输性能是可行的。

\section{本章小结}
本章主要介绍了与网络功能虚拟化技术已经本课题相关的一些技术背景。首先,本章简单总结了下网络功能虚拟化的发展历史和当前的发展现状,介绍了网络功能虚拟化出现的背景,包括与网络功能虚拟化实现相关的虚拟化技术以及多核服务器的架构背景。本章着重介绍了本研究所基于NFV实际应用平台Clearwater和IMS服务,包括其业务功能简介和基本架构介绍。最后本章通过多核服务器的性能观测实验说明了在多核服务器中存在的实际性能问题,为接下来的模型和设计提供必要支撑。