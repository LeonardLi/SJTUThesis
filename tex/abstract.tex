%# -*- coding: utf-8-unix -*-
%%==================================================
%% abstract.tex for SJTU Master Thesis
%%==================================================

\begin{abstract}
网络功能虚拟化 (NFV) 利用虚拟化的基础设施展现了业界所期待的敏捷性和可扩展性,但是软件化网络服务的性能仍然是阻碍其大规模普及应用的瓶颈之一。与传统虚拟化应用不同,大多数网络功能虚拟服务是多个虚拟机 (VMs) 以服务链 (SFC) 的形式运行在标准的大容量服务器中。每台虚拟机中运行着专用的虚拟网络功能 (VNF),现有的虚拟机监视器 (Hypervisor) 仍然将这些虚拟机当做单独的虚拟机实例来为分配资源和维护。当一条服务链被请求服务时,现有的虚拟化资源管理缺少对每个所需要的VNF(VM)之间逻辑连接的识别,单独地为每个虚拟机分配物理资源。这样的资源分配方式会由于多核物理机底层架构的原因导致组成的服务链的性能波动,如何提升大容量多核服务器的资源利用效率成为了一个挑战。

为了解决这个问题,本研究提出了针对服务链的资源抽象模型和资源分配算法,结合真实的IP多媒体子系统Clearwater服务,实现了本文的优化框架原型。通过采样底层服务的性能信息作为模型和算法的输入,根据生成的映射策略对实际参与的服务网络功能实例进行组链,完成所需服务的组建。

最后,在实验室的真实网络环境下对本设计与默认的资源分配方法进行对比和性能评估。数据分析表明,本设计可以减少平均40\%的服务延迟,并提升接近3倍的网络吞吐。通过这些实验数据可以证明,本文所提出的优化原型针在Clearwater服务背景下性能有了明显的提升。

\keywords{\large 网络功能虚拟化 \quad 服务链 \quad 多核服务器 \quad}
\end{abstract}

\begin{englishabstract}
Network Function Virtualization (NFV) shows agility and scalability in industry when deployed in cloud computing infrastructure, but the performance of softwarized network service still impedes its popularization. Most of the network service in cloud run as service function chain (SFC) in Standard High Volume Server, consisting of virtual machines (VMs) in tandem. Each VM runs a dedicated virtual network function (VNF) in SFC. The hyper-visor treats these VM with running VNF as single instance to maintain. When a certain SFC is called for serving, the resource manager lacks the recognition of details on the connections between running VFs and separately allocate the resource which consequently lead to low efficient resource utilization for SHVS. This kind of resource allocation will result in performance fluctuation in SFC. How to optimize the SHVS resource utilization for chain-style NFV application poses a great challenge for us.

To resolve this problem, in this paper, we propose a new resource abstraction considering the chain-style NFV workload with underlying characteristics and design a greedy algorithm based on our model to map resource. We implement our framework in real IP Multimedia Subsystem (IMS) platform. 

At last, the framework is tested and compared to existing generic methods. By the examination statistics, our framework can decrease 40\% average IMS service latency and improve the throughput within the SFC up to 300\%.

\englishkeywords{NFV, Service Function Chain, Multi-core Server}
\end{englishabstract}

